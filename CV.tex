% Documentation an template for the NU-Math Vita class Version 0.4
% Miguel A. Lerma (mlerma@math.northwestern.edu) - 2005/08/23
%
% Process it with the command ``latex filename''.
%
%
% The document starts with
% \documentclass{nuvita}
%
% or
%
% \documentclass[options]{nuvita}
%
% Possible options are: 10pt, 11pt, 12pt.
% For instance:
\documentclass[10pt,letterpaper]{moderncv}

% moderncv themes
\moderncvtheme[blue]{classic}                 % optional argument are 'blue' (default), 'orange', 'red', 'green', 'grey' and 'roman' (for roman fonts, instead of sans serif fonts)
\usepackage{footmisc}
%\moderncvtheme[green]{classic}                % idem

% character encoding
\usepackage[utf8]{inputenc}                   % replace by the encoding you are using
\usepackage{url}
\usepackage{graphicx}
%\usepackage{multibib}
\usepackage{lastpage}
%\usepackage{hyperref}
\usepackage{comment}
\rfoot{\addressfont\itshape\textcolor{gray}{Page \thepage\ of \pageref{LastPage}}}

%\newcites{papers,talks,posters}{{Paper},{Invited Talks},{Posters}}

% adjust the page margins
\usepackage[scale=0.9]{geometry}
%\usepackage[sorting=ynt,backend=biber]{biblatex}
%\bibliography{CV}

\recomputelengths                             % required when changes are made to page layout lengths
%%%%%%%%%%%%%%%%%%%% PERSONAL INFORMATION %%%%%%%%%%%%%%%%%%%

% Here we enter some personal information.

% % personal data
\name{Fatemeh}{Almodaresi}

%\title{CV}               % optional, remove the line if not wanted

\address{MaRS Centre}{661 University Avenue, Suite 510}{Toronto, Ontario Canada, M5G 0A3}

%\phone{(548) -- 333 -- 1386}                    % optional, remove the line if not wanted

%\phone{phone (optional)}                      % optional, remove the line if not wanted
%\fax{fax (optional)}                          % optional, remove the line if not wanted

%\email{falmodar@cs.umd.edu}                      % optional, remove the line if not wanted
\email{falmodaresi@oicr.on.ca}                      % optional, remove the line if not wanted
\extrainfo{\url{https://www.linkedin.com/in/fataltes}\\
  \url{https://github.com/fataltes}\\
  \url{https://fataltes.github.io}}
%\photo[64pt]{images/me.png}

%quote{Some quote (optional)}                 % optional, remove the line if not wanted

\renewcommand*{\bibliographyitemlabel}{[\arabic{enumiv}]}
%%%%%%%%%%%%%%%%%%%%%%%%%% DOCUMENT %%%%%%%%%%%%%%%%%%%%%%%%%%
% Here the document begins.
\begin{document}
\maketitle

%%%%%%%%%%%%%%%%%%% PREDEFINED ENVIRONMENTS %%%%%%%%%%%%%%%%%%


% The predefined environments print a large boldfaced title
% (e.g.: ``Education:'') and arrange the information as a
% table or list.

%\begin{goals}
%My goals are what they are...
%\end{goals}

\section{Research Interests}
\cvline{}{Computational Biology}
\cvline{}{Probabilistic Modeling and Machine Learning}
\cvline{}{Algorithms and Data Structures}
%\cvline{}{Natural Language Processing}


\section{Skills}
\cventry{Programming Languages}{C++ (expert), Python (expert), R (familiar), Java Core(expert), MATLAB (familiar), NetLogo (familiar), C\# (familiar), Rust (starter), MySQL (expert), Oracle (expert)}{}{}{}{}
\cventry{Tools, Libraries and Frameworks}{Bioinformatics Analysis Tools (e.g. SAMtools, Bedtools, Picard, BWA, BWA-MEM2, STAR, STARsolo, Bowtie2, Salmon, Minimap2, Seurat, IGV, StringTie2, DESeq2, etc.), LLM (e.g. LangChain framework), Popular Python Libraries (numpy, pandas, scipy.stats, sklearn, matplotlib, and seaborn), ShinyR}{}{}{}{}
\cventry{Databases}{TCGA, PCAWG, GO ontology and enrichment analysis tools, Human Cell Atlas, Reactome}{}{}{}{}
\cventry{Machine Learning}{Large-Language-Models (LLMs), feature selection, dimensionality reduction, (un)supervised learning, pattern recognition}{}{}{}{}
\cventry{Others}{Git, Bash script, Snakemake, Docker, Anaconda Platform, Jupyter Notebook, IntelliJ IDEA, CLion, Pycharm, Atlassian Jira}{}{}{}{}




% Education environment. It is a table with fields separated by
% andpersands (&), and lines separated by double backslashes (\\).
\section{Education}
%\cventry{}{}{}{}{}{}
%GPA \textit{3.94}
%GPA \textit{17.75/20}
%GPA \textit{16.07/20}
\cventry{Feb 2021-Now}{PostDoctoral Fellow}{Adaptive Oncology Department, Ontario Institute for Cancer Research (\httplink[OICR]{https://oicr.on.ca/})}{ON, CAN}{}{ Advisor -- Prof. Lincoln Stein}
\cventry{2019-Aug 2020}{Ph.D}{Computer Science Department, University of Maryland (\httplink[UMD]{https://www.cs.umd.edu/})}{MD, US}{}{ Advisor -- Prof. Rob Patro}
\cventry{2015-2019}{MS}{Computer Science Department, Stony Brook University (\httplink[SBU]{https://www.cs.stonybrook.edu/})}{NY, US}{}{ Advisor -- Prof. Rob Patro}
\cventry{2004-2009}{BS}{School of Electrical \& Computer Engineering (\httplink[ECE]{http://ece.ut.ac.ir/en?destination=slide}), University of Tehran}{Tehran, Iran}{}{}

\section{Selected Research Projects}

\cventry{May. 2023 - Present}{PathwayLLM}{``Talk to your Pathway''}{A collaboration between Adaptive Oncology Dep., OICR and WangLab., Vector Institute}
{}{Leading a team in developing a computational solution using \textbf{advanced language models}, \textbf{Reactome Knowledge Graph}, and \textbf{PubMed data} through text mining and embedding techniques. Our system employs retrieval-augmented generation (\textbf{RAG}), and utilizes \textbf{vector databases such as Pinecone} for text indexing, \textbf{LLM embeddings such as Huggingface embeddings} and \textbf{prompt engineering techniques}. We encorporated manual reference construction and fact checking through text semantic similarity search between results and source documents. It generates Reactome-like pathway summaries, handles complex pathway queries, and formulates hypotheses for uncharted pathways. Ongoing work includes expanding to more databases and optimizing performance.}
%{Leading a team of graduate students to engineer a computational solution employing advanced language models to address three core research objectives: generating Reactome-caliber pathway summaries for non-Reactome instances, constructing an efficient retrieval-augmented generation (RAG) system for precise pathway inquiries, and devising hypotheses for uncharted pathways. Leveraging PubMed data and RAG techniques, the project harnesses pathway knowledge graphs originating from BioPax files, Google Search scraping, and Reactome-referenced papers. Through orchestrated sequential method chains, the developed chatbot adeptly tackles intricate pathway queries, and produces succinct summaries. Ongoing work encompasses diversifying to additional pathway databases, imbuing predictive capabilities such as formulating hypotheses for potentially novel pathways, and optimizing performance via advanced indexing tools while enriching knowledge assimilation.}

%\cventry{Nov. 2022 - Present}{Ultra-small ultra-fast accurate scBERT using biological priors}{}{Adaptive Oncology Dep., OICR}
%{}{scBERT which has recently been published is a slightly modified and adjusted BERT that provides an accurate model for representing the single cell gene expressions and uses it for downstream task of cell annotation. However, the presentation has a few flaws that stops the model from easily extending to big single cell ATLASes. It is also not tuned specifically for unsupervised tasks such as data imputation which are much harder in nature. The goal of this project is to modify the model and map the input to a more meaningful representation of the cells and genes. The new presentation would allow for much smaller inputs and therefore save memory and training time. We also use biological priors such as pathways, GRNs, and GO to enrich the model. The input would consider genes as words and cells as sentences in the mapping of biological data to NLP and have additional embeddings derived from biological knowledge. In addition to that we want to provide an interpretation of the model to use for imputation purposes. A new line of research is growing regarding interpretation of the transformer models which would allow us to have a better understanding of the underlying gene2gene interactions that results in a highly accurate prediction of cell types (or other annotations) using transformers.}

\cventry{Oct. 2022 - Present}{ScReps}{``Discovering novel cancer signals from retroposans in single cell unaligned reads''}{Adaptive Oncology Dep., OICR}
{}{In this project, I explore the therapeutic potential of retroelement misexpression in cancer cells for cancer prognosis. The computational pipeline involves multiple key steps, including 
\begin{itemize}
\item Aligning 10X reads with Cellranger 
\item Integrating samples using Harmony
\item Annotating cells via Seurat
\item Selecting unaligned reads
\item Aligning these reads to Repbase repetitive elements database using Salmon
\item and Conducting single-cell differential expression analysis between Healthy and Condition samples for each cell type using ZINB\_DESeq2 and AggregateBioVar3. 
\end{itemize}
Applying the data on diverse datasets, including control datasets like 10X-PBMC8k, Medulloblastoma Tumor Cells (G3, G4, SHH, CPA, PFA-PFB), and Sarcoma Immune Cells (across various conditions and individuals). I found significant expression-based signals differentiating cancer and healthy samples. The pipeline is written in bash. Next steps are creating a Snakemake for the pipeline and expanding initial analysis to more datasets to confirm the results.}

\cventry{Feb. 2021 - Present}{Immunotherapeutic Targeting of the U1 snRNA Mutation in Cancer}{}{Adaptive Oncology Dep., OICR}
{}{I led a pioneering project across multiple institutions targeting novel U1 snRNA mutations prevalent in hard-to-treat cancers like CLL and pediatric cerebellar MB. Our pipeline, starting from Nanopore long reads, involved rigorous sequence filtering, read assembly, and transcript consolidation. We first dscard sequences with imbalance-primer using Pychopper and then utilized advanced tools such as minimap2, and StringTie2 for alignment and assembly of the rest of the reads. With access to both Mutant and Wildtype samples, we filtered out potential sequencing artifacts using paired QC-based analysis. By focusing on mutant-specific transcripts and selecting key MHC-binding peptides, we aimed to develop targeted therapeutics for these U1 mutant tumors, offering a promising approach to improve treatment outcomes in challenging cancer cases.}
%{
%The goal of this project is to Explore targeted therapeutics against a novel class of mutation in several types of hard-to-treat cancers. The mutation happens in the U1 snRNA and is present in majority of several cancer types such as CLL (chronic lymphocytic leukemia) and pediatric cerebellar MB (medulloblastoma). Since the patterns of mis-splicing caused by this mutation are identical across different samples, this makes an opportunity for having a global therapeutic approach towards neo-antigens present in U1 mutant tumors.
%For that matter we have explored the primaries as well as the cell lines for both CLL and MB samples and built up a pipeline for finding interesting peptides that is highly present across a great proportion of samples. The pipeline's focus is on predicting and validating MHC class I neo-peptides (neo-MAPs).
%}

\cventry{Aug. 2017 - 2021}{Mantis}{``A fast, small, and exact large-scale sequence-search index''}{Computational Biology Lab., SBU, UMD}
{\newline{} \url{https://github.com/splatlab/mantis/tree/mergeMSTs}}{Mantis is a space and time efficient data structure to index and query large collections of raw sequencing read experiments.
The index is based on colored de Bruijn graph representation and therefore supports graph-based operations such as graph traversal, and bubble calling useful for assembly and variation detection. In our recent work we have advanced the index to more than 30,000 raw read sequencing samples and enabled the nice feature of gradual growth by making the index incrementally updatable.}
\cventry{Jun. 2017 - Aug. 2020}{Pufferfish and Puffaligner}{``A space and time-efficient compacted de Bruijn graph index and aligner''}{Computational Biology Lab., SBU, UMD}
{\newline{} \url{https://github.com/COMBINE-lab/pufferfish/tree/develop}}{Pufferfish is an efficient data structure for indexing colored compacted de Bruijn graphs. This index achieves a balance between time and space resources by making use of succinct data structures and minimum perfect hash function. PuffAligner, our recent work, is a highly sensitive aligner on top of Pufferfish for aligning different types of short sequencing reads to a huge population of references, specifically good in the representation of high similarity in the reference sequences.}
\cventry{Apr. - Jun. 2017}{Rainbowfish}{``A succinct colored de Bruijn graph data structure''}{Computational Biology Lab., SBU}{\newline{} \url{https://github.com/COMBINE-lab/rainbowfish}}{This tool provides a new data structure to store and query colored de Bruijn graphs that in case of large data sets improves storage by more than twenty times compared to state-of-the-art tools without hurting performance of the queries.}
\cventry{Nov. 2016 - Apr. 2018}{Grouper}{An extension to ``Rapid Clustering'' tool}{Computational Biology Lab., SBU}{\newline{} \url{https://github.com/COMBINE-lab/grouper}}{Grouper is a tool for clustering contigs of a de novo transcriptome assembly. We improved the accuracy of clustering by making use of orphan reads, for which each end of the pair is mapped to a different reference sequence.}
\cventry{Aug 2016-Jan 2017}{MLDD}{``Multi-Level Distribution Detection''}{Data Science Lab., SBU}{}{Using statistical tests and classification models such as NaiveBayes we show how distribution of NLP features in social media changes in different levels of analysis (county, user, and message). This can highly affect prior assumptions for further text analysis as we show that central-limit theorem could be applied in social media language analysis as well.}
\cventry{2013-2014}{AutismFD}{``A game to improve face emotion detection in children with Autism''}{}{}{Beside collaboration with psychology students to design the method, I also implemented the idea as a tool in C\# language. This package was used in a treatment center to help children with Autism to identify face emotions and track their progress over time.}
%\cventry{2012-2013}{PersonalityMatcher}{``Improve students’ educational condition via detecting best friendship communities'' project}{IUST}{\newline{}Under the supervision of Professors Jahed \& Mozayani}{We designed and developed a multi-agent system in Netlogo environment based on NEO personality questionnaire.}

%\section{Publications}
%% Publications %%
%\nocitepapers{}
\nocite{*}
%\nocite{
%mzakeri:2017:factorization,
%falmodaresit:2017:mldd,
%falmodaresit:2014:personality
%}
%\bibliographystylepapers{plainyr-rev}
%\bibliographypapers{papers}
\bibliographystyle{plainyr-rev}
\bibliography{papers}

%\nocitetalks{}
%\bibliographystyletalks{plainyr-rev}
%\bibliographytalks{talks}

%\nociteposters{}
%\bibliographystyleposters{plainyr-rev}
%\bibliographyposters{posters}


\section{Work Experiences}
\cventry{Jun-Aug 2016}{Member of the NLP Team}{Third Frederick Jelinek Memorial Summer Workshop (\httplink[JSALT]{http://www.clsp.jhu.edu/workshops/16-workshop/})}{MD, US}{}{JSALT is a well-known summer workshop in Language and Speech organized by JHU at Baltimore each year.
.\newline{}%
During the project, we worked on analyzing and forecasting social media user’s psychological state based on their language in their posts using statistical methods such as significance tests and time series models such as ARMA and ARIMA.}
\cventry{Jan-Aug 2015}{Senior Designer and Developer}{\httplink[Nexeven AB]{http://www.nexeven.com/}}{Tehran, Iran}{}{Nexeven AB is a Swedish company and a niche player in the online video broadcasting field.}
\cventry{2011-2015}{Team Supervisor, Senior Designer and Developer}{Tosan Intelligent Data Miners Co. (\httplink[TIDM]{http://www.tosanidm.ir/})}{Data Mining Development Team}{Tehran, Iran}{TIDM is the first solution provider for fraud detection and anti-money laundry in banking section in Iran, a Subsidiary of Tosan Company.
%\newline{}%
\begin{itemize}%
\item \textbf{Customer Relationship Management System} [2014] \\
In this project we use statistical and data mining methods to calculate customer`s RFM, CLV, and churn probability.
\item \textbf{Data mining Module, Operational Intelligence System} [2014] \\
This module, developed in PLSQL, uses Statistical and Mining Methods such as regression models, error functions, k-means, and SVM to detect fraudulent transactions online in the stream of transactions.
\item \textbf{Customer Name Similarity Detection Module} [2013] \\
As a part of Anti-money Laundry System, this module uses natural language algorithms to detect accounts with similar names. The whole system is developed in PLSQL and now operational in many private banks in Iran including \httplink[Eghtesad-Novin]{http://english.enbank.ir/} and Ansar Bank.
\item \textbf{Unsupervised Fraud Detection System, Version 1 \& 2} [2011-2014] \\
Version 1 which is fully designed and developed by myself is now operational in \httplink[Saman]{http://sb24.ir/En/Business/index.html} Bank, \httplink[Ansar]{http://www.ansarbank.com/EnHome.aspx} Bank, and \httplink[Mehr-e-Eghtesad]{http://www.mebank.ir/} Bank in Iran. Version 2 is now installed in \httplink[Eghtesad-Novin]{http://english.enbank.ir/} Bank.
\end{itemize}}
%\cventry{2009-2011}{Java and UI Developer }{\httplink[Tosan]{http://tosan.com/en/default.aspx} Co}{Tehran, Iran}{}{Tosan Company is a pioneer company for total banking solutions with more than fifteen Iranian financial institutes in its customer list. As a member of a team of nearly 20 people, I participated in developing the UI of Internet banking system.}

\section{Honors \& Awards}
\cventry{2020}{Larry Davis Dissertation Award}{UMD}{}{}{}
\cventry{2019}{Grace Hopper Conference 2019 (GHC19) Scholarship}{}{}{}{}
\cventry{2019}{Catacosinos Fellowship for Excellence in Computer Science}{SBU}{}{}{}
\cventry{2019}{RECOMB2019 Conference Travel Fellowship}{}{}{}{}
\cventry{2018}{ISMB2018 Conference Travel Fellowship}{}{}{}{}
\cventry{2016}{CS Department Best TA Award}{SBU}{}{}{}
\cventry{2015}{Special CS Department Chair Fellowship}{SBU}{}{}{}

% \section{\printbibheading}
% \printbibliography[heading=subbibliography]

% \fullcite{phillippy:2009:dsim}\\
% \fullcite{Sankaranarayanan:2009:fview}\\
% \fullcite{patro:2009:sf-detection}\\
% \fullcite{patro:2009:sg-keyframe}\\
% \fullcite{maximo:2010:sample}\\
% \fullcite{patro:2010:social-snapshot}\\
% \fullcite{patro:2010:m3}\\

\section{Teaching Experiences}
\cventry{Fall 2017}{Teaching Assistant}{Computational Biology}{Stony Brook University}{}{}
\cventry{Spring 2017}{Teaching Assistant}{Machine Learning}{Stony Brook University}{}{}
%\cventry{2013}{Teacher}{C++ Programming Language}{Farzanegan High School [\httplink[NODET]{http://en.wikipedia.org/wiki/National_Organization_for_Development_of_Exceptional_Talents}]}{}{}
%\cventry{2013}{Teacher}{Developing simple motion detection algorithms in MATLAB}{Farzanegan High School [\httplink[NODET]{http://en.wikipedia.org/wiki/National_Organization_for_Development_of_Exceptional_Talents}]}{}{}
%\cventry{Fall 2008}{Teaching Assistant}{Artificial intelligence}{University of Tehran}{}{}

% Employment environment. It is a description list.
% Use ``\bl'' to break a line.

%\section{Extracurricular Activities \& Interests}
%\cventry{2010-present}{Playing Classical Guitar}{}{}{}{}
%\cventry{2002}{Organized a Provincial Seminar about Women, their Activities, and Prospers with Collaboration of Yazd Governor}{}{Yazd, Iran}{}{}
%\cventry{2001-2004}{Playing Basketball}{Awarded Second Rank in Yazd Province}{Yazd, Iran}{}{}
\end{document}
