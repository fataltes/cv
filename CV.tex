% Documentation an template for the NU-Math Vita class Version 0.4
% Miguel A. Lerma (mlerma@math.northwestern.edu) - 2005/08/23
%
% Process it with the command ``latex filename''.
%
%
% The document starts with
% \documentclass{nuvita}
%
% or
%
% \documentclass[options]{nuvita}
%
% Possible options are: 10pt, 11pt, 12pt.
% For instance:
\documentclass[10pt,letterpaper]{moderncv}

% moderncv themes
\moderncvtheme[blue]{classic}                 % optional argument are 'blue' (default), 'orange', 'red', 'green', 'grey' and 'roman' (for roman fonts, instead of sans serif fonts)
\usepackage{footmisc}
%\moderncvtheme[green]{classic}                % idem

% character encoding
\usepackage[utf8]{inputenc}                   % replace by the encoding you are using
\usepackage{url}
\usepackage{graphicx}
%\usepackage{multibib}
\usepackage{lastpage}
%\usepackage{hyperref}
\usepackage{comment}
\rfoot{\addressfont\itshape\textcolor{gray}{Page \thepage\ of \pageref{LastPage}}}

%\newcites{papers,talks,posters}{{Paper},{Invited Talks},{Posters}}

% adjust the page margins
\usepackage[scale=0.9]{geometry}
%\usepackage[sorting=ynt,backend=biber]{biblatex}
%\bibliography{CV}

\recomputelengths                             % required when changes are made to page layout lengths
%%%%%%%%%%%%%%%%%%%% PERSONAL INFORMATION %%%%%%%%%%%%%%%%%%%

% Here we enter some personal information.

% % personal data
\name{Fatemeh}{Almodaresi}

%\title{CV}               % optional, remove the line if not wanted

\address{244 New Computer Science Building}{Stony Brook University}{Stony Brook, NY 11794}

\phone{(631) -- 974 -- 8156}                    % optional, remove the line if not wanted

%\phone{phone (optional)}                      % optional, remove the line if not wanted
%\fax{fax (optional)}                          % optional, remove the line if not wanted

\email{falmodaresit@cs.stonybrook.edu}                      % optional, remove the line if not wanted
\extrainfo{\url{https://www.linkedin.com/in/fataltes}\\
  \url{https://github.com/fataltes}\\
  \url{https://fataltes.github.io}}
%\photo[64pt]{images/me.png}

%quote{Some quote (optional)}                 % optional, remove the line if not wanted

\renewcommand*{\bibliographyitemlabel}{[\arabic{enumiv}]}
%%%%%%%%%%%%%%%%%%%%%%%%%% DOCUMENT %%%%%%%%%%%%%%%%%%%%%%%%%%
% Here the document begins.
\begin{document}
\maketitle

%%%%%%%%%%%%%%%%%%% PREDEFINED ENVIRONMENTS %%%%%%%%%%%%%%%%%%


% The predefined environments print a large boldfaced title
% (e.g.: ``Education:'') and arrange the information as a
% table or list.

%\begin{goals}
%My goals are what they are...
%\end{goals}

\section{Research Interests}
\cvline{}{Computational Biology}
\cvline{}{Data Mining, Pattern Recognition, and Data Analysis}
%\cvline{}{Natural Language Processing}
% Education environment. It is a table with fields separated by
% andpersands (&), and lines separated by double backslashes (\\).
\section{Education}
%\cventry{}{}{}{}{}{}
%GPA \textit{3.94}
%GPA \textit{17.75/20}
%GPA \textit{16.07/20}
\cventry{2015-present}{Ph.D}{Computer Science Department, Stony Brook University (\httplink[SBU]{https://www.cs.stonybrook.edu/})}{NY, USA}{}{ Advisor -- Prof. Rob Patro}
\cventry{2009-2011}{MS}{School of Computer Engineering, Iran University of Science and Technology (\httplink[IUST]{http://www.iust.ac.ir/index.php?sid=14&slc_lang=en})}{Tehran, Iran}{}{ Advisor -- Prof. Jahed Motlagh}
\cventry{2004-2009}{BS}{School of Electrical \& Computer Engineering (\httplink[ECE]{http://ece.ut.ac.ir/en?destination=slide}), University of Tehran}{Tehran, Iran}{}{}

\section{Selected Research Projects}
\cventry{Aug. 2017 - Present}{Mantis}{``A fast, small, and exact large-scale sequence-search index''}{Computational Biology Lab., SBU}
{\newline{} \url{https://github.com/splatlab/mantis}}{Mantis is a space and time efficient data structure to index and query large collections of raw sequencing read experiments. The index is based on colored de Bruijn graph representation and therefore supports graph traversal, bubble calling and other graph-based computational and biological analysis.}
\cventry{Jun. 2017 - Jan. 2018}{Pufferfish}{``A space and time-efficient compacted de Bruijn graph index''}{Computational Biology Lab., SBU}
{\newline{} \url{https://github.com/COMBINE-lab/pufferfish}}{Pufferfish is an efficient data structure for indexing colored compacted de Bruijn graphs. This tool can achieve a balance between time and space resources by making use of succinct data structures and minimum perfect hash function. Pufferfish provides the underlying data structure for mapping short sequencing reads to a huge population of references while keeping the mapping information for each reference individually. Plan to submit to Recomb 2018.}
\cventry{Apr. - Jun. 2017}{Rainbowfish}{``A succinct colored de Bruijn graph data structure''}{Computational Biology Lab., SBU}{\newline{} \url{https://github.com/COMBINE-lab/rainbowfish}}{This tool provides a new data structure to store and query colored de Bruijn graphs that in case of large data sets improves storage by more than twenty times compared to state-of-the-art tools without hurting performance of the queries.}
\cventry{Nov. 2016 - Present}{Grouper}{An extension to ``Rapid Clustering'' tool}{Computational Biology Lab., SBU}{\newline{} \url{https://github.com/COMBINE-lab/grouper}}{Grouper is a tool for clustering contigs of a de novo transcriptome assembly. We improved the accuracy of clustering by making use of orphan reads, for which each end of the pair is mapped to a different reference sequence (accepted to Bioinformatics).}
\cventry{Aug 2016-Jan 2017}{MLDD}{``Multi-Level Distribution Detection''}{Data Science Lab., SBU}{}{Using statistical tests and classification models such as NaiveBayes we show how distribution of NLP features in social media changes in different levels of analysis (county, user, and message). This can highly affect prior assumptions for further text analysis as we show that central-limit theorem could be applied in social media language analysis as well.}
\cventry{2013-2014}{AutismFD}{``A game to improve face emotion detection in children with Autism''}{}{}{Beside collaboration with psychology students to design the method, I also implemented the idea as a tool in C\# language. This package was used in a treatment center to help children with Autism to identify face emotions and track their progress over time.}
%\cventry{2012-2013}{PersonalityMatcher}{``Improve students’ educational condition via detecting best friendship communities'' project}{IUST}{\newline{}Under the supervision of Professors Jahed \& Mozayani}{We designed and developed a multi-agent system in Netlogo environment based on NEO personality questionnaire.}

%\section{Publications}
%% Publications %%
%\nocitepapers{}
\nocite{*}
%\nocite{
%mzakeri:2017:factorization,
%falmodaresit:2017:mldd,
%falmodaresit:2014:personality
%}
%\bibliographystylepapers{plainyr-rev}
%\bibliographypapers{papers}
\bibliographystyle{plainyr-rev}
\bibliography{papers}

%\nocitetalks{}
%\bibliographystyletalks{plainyr-rev}
%\bibliographytalks{talks}

%\nociteposters{}
%\bibliographystyleposters{plainyr-rev}
%\bibliographyposters{posters}


\section{Work Experiences}
\cventry{Jun-Aug 2016}{Member of the NLP Team}{Third Frederick Jelinek Memorial Summer Workshop (\httplink[JSALT]{http://www.clsp.jhu.edu/workshops/16-workshop/})}{Baltimore}{}{JSALT is a well-known summer workshop in Language and Speech organized by JHU each year.
.\newline{}%
During the project, we worked on analyzing and forecasting social media user’s psychological state based on their language in their posts using statistical methods such as significance tests and time series models such as ARMA and ARIMA.}
\cventry{Jan-Aug 2015}{Senior Designer and Developer}{\httplink[Nexeven AB]{http://www.nexeven.com/}}{}{}{Nexeven AB is a Swedish company and a niche player in the online video broadcasting field.}
\cventry{2011-2015}{Team Supervisor, Senior Designer and Developer}{Tosan Intelligent Data Miners Co. (\httplink[TIDM]{http://www.tosanidm.ir/})}{}{Data Mining Development Team}{TIDM is the first solution provider for fraud detection and anti-money laundry in banking section in Iran, a Subsidiary of Tosan Company.
%\newline{}%
\begin{itemize}%
\item \textbf{Customer Relationship Management System} [2014] \\
In this project we use statistical and data mining methods to calculate customer`s RFM, CLV, and churn probability.
\item \textbf{Data mining Module, Operational Intelligence System} [2014] \\
This module, developed in PLSQL, uses Statistical and Mining Methods such as regression models, error functions, k-means, and SVM to detect fraudulent transactions online in the stream of transactions.
\item \textbf{Customer Name Similarity Detection Module} [2013] \\
As a part of Anti-money Laundry System, this module uses natural language algorithms to detect accounts with similar names. The whole system is developed in PLSQL and now operational in many private banks in Iran including \httplink[Eghtesad-Novin]{http://english.enbank.ir/} and Ansar Bank.
\item \textbf{Unsupervised Fraud Detection System, Version 1 \& 2} [2011-2014] \\
Version 1 which is fully designed and developed by myself is now operational in \httplink[Saman]{http://sb24.ir/En/Business/index.html} Bank, \httplink[Ansar]{http://www.ansarbank.com/EnHome.aspx} Bank, and \httplink[Mehr-e-Eghtesad]{http://www.mebank.ir/} Bank in Iran. Version 2 is now installed in \httplink[Eghtesad-Novin]{http://english.enbank.ir/} Bank.
\end{itemize}}
\cventry{2009-2011}{Java and UI Developer }{\httplink[Tosan]{http://tosan.com/en/default.aspx} Co}{}{}{Tosan Company is a pioneer company for total banking solutions with more than fifteen Iranian financial institutes in its customer list. As a member of a team of nearly 20 people, I participated in developing the UI of Internet banking system.}

\section{Honors \& Awards}
\cventry{2016}{CS Department Best TA Award}{}{}{}{}
\cventry{2015}{Special CS Department Chair Fellowship}{}{}{}{}

% \section{\printbibheading}
% \printbibliography[heading=subbibliography]

% \fullcite{phillippy:2009:dsim}\\
% \fullcite{Sankaranarayanan:2009:fview}\\
% \fullcite{patro:2009:sf-detection}\\
% \fullcite{patro:2009:sg-keyframe}\\
% \fullcite{maximo:2010:sample}\\
% \fullcite{patro:2010:social-snapshot}\\
% \fullcite{patro:2010:m3}\\

\section{Teaching Experiences}
\cventry{Fall 2017}{Teaching Assistant}{Computational Biology}{Stony Brook University}{}{}
\cventry{Spring 2017}{Teaching Assistant}{Machine Learning}{Stony Brook University}{}{}
\cventry{2013}{Teacher}{C++ Programming Language}{Farzanegan High School [\httplink[NODET]{http://en.wikipedia.org/wiki/National_Organization_for_Development_of_Exceptional_Talents}]}{}{}
\cventry{2013}{Teacher}{Developing simple motion detection algorithms in MATLAB}{Farzanegan High School [\httplink[NODET]{http://en.wikipedia.org/wiki/National_Organization_for_Development_of_Exceptional_Talents}]}{}{}
\cventry{Fall 2008}{Teaching Assistant}{Artificial intelligence}{University of Tehran}{}{}

% Employment environment. It is a description list.
% Use ``\bl'' to break a line.

\section{Skills}
\cventry{Programming Languages}{Python (expert), Java Core(expert), C++ (familiar), R (familiar), MATLAB (familiar), NetLogo (familiar), C\# (familiar)}{}{}{}{}
\cventry{Libraries and Frameworks}{Popular Python Libraries (numpy, pandas, scipy.stats, sklearn), ShinyR, Spring Framework, Hibernate, Play Framework}{}{}{}{}
\cventry{Databases}{Oracle (expert), MySQL (expert), MongoDB (familiar)}{}{}{}{}
\cventry{Other Tools}{Git, Atlassian Jira, Atlassian Confluence, ThoughtWorks Go, Anaconda Platform, Pycharm, Jupyter Notebook, IntelliJ IDEA}{}{}{}{}

%\section{Extracurricular Activities \& Interests}
%\cventry{2010-present}{Playing Classical Guitar}{}{}{}{}
%\cventry{2002}{Organized a Provincial Seminar about Women, their Activities, and Prospers with Collaboration of Yazd Governor}{}{Yazd, Iran}{}{}
%\cventry{2001-2004}{Playing Basketball}{Awarded Second Rank in Yazd Province}{Yazd, Iran}{}{}
\end{document}
